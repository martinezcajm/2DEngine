\href{https://github.com/nlohmann/json/releases}{\tt }

\href{https://travis-ci.org/nlohmann/json}{\tt } \href{https://ci.appveyor.com/project/nlohmann/json}{\tt } \href{https://coveralls.io/r/nlohmann/json}{\tt } \href{https://scan.coverity.com/projects/nlohmann-json}{\tt } \href{https://www.codacy.com/app/nlohmann/json?utm_source=github.com&amp;utm_medium=referral&amp;utm_content=nlohmann/json&amp;utm_campaign=Badge_Grade}{\tt } \href{http://melpon.org/wandbox/permlink/nv9fOg0XVVhWmFFy}{\tt } \href{http://nlohmann.github.io/json}{\tt } \href{https://raw.githubusercontent.com/nlohmann/json/master/LICENSE.MIT}{\tt } \href{https://github.com/nlohmann/json/releases}{\tt } \href{http://github.com/nlohmann/json/issues}{\tt } \href{http://isitmaintained.com/project/nlohmann/json}{\tt } \href{https://bestpractices.coreinfrastructure.org/projects/289}{\tt }


\begin{DoxyItemize}
\item \href{#design-goals}{\tt Design goals}
\item \href{#integration}{\tt Integration}
\item \href{#examples}{\tt Examples}
\begin{DoxyItemize}
\item \href{#json-as-first-class-data-type}{\tt J\+S\+ON as first-\/class data type}
\item \href{#serialization--deserialization}{\tt Serialization / Deserialization}
\item \href{#stl-like-access}{\tt S\+T\+L-\/like access}
\item \href{#conversion-from-stl-containers}{\tt Conversion from S\+TL containers}
\item \href{#json-pointer-and-json-patch}{\tt J\+S\+ON Pointer and J\+S\+ON Patch}
\item \href{#implicit-conversions}{\tt Implicit conversions}
\item \href{#arbitrary-types-conversions}{\tt Conversions to/from arbitrary types}
\item \href{#binary-formats-cbor-and-messagepack}{\tt Binary formats (C\+B\+OR and Message\+Pack)}
\end{DoxyItemize}
\item \href{#supported-compilers}{\tt Supported compilers}
\item \href{#license}{\tt License}
\item \href{#contact}{\tt Contact}
\item \href{#thanks}{\tt Thanks}
\item \href{#used-third-party-tools}{\tt Used third-\/party tools}
\item \href{#projects-using-json-for-modern-c}{\tt Projects using J\+S\+ON for Modern C++}
\item \href{#notes}{\tt Notes}
\item \href{#execute-unit-tests}{\tt Execute unit tests}
\end{DoxyItemize}

\subsection*{Design goals}

There are myriads of \href{http://json.org}{\tt J\+S\+ON} libraries out there, and each may even have its reason to exist. Our class had these design goals\+:


\begin{DoxyItemize}
\item {\bfseries Intuitive syntax}. In languages such as Python, J\+S\+ON feels like a first class data type. We used all the operator magic of modern C++ to achieve the same feeling in your code. Check out the \href{#examples}{\tt examples below} and you\textquotesingle{}ll know what I mean.
\item {\bfseries Trivial integration}. Our whole code consists of a single header file \href{https://github.com/nlohmann/json/blob/develop/src/json.hpp}{\tt {\ttfamily json.\+hpp}}. That\textquotesingle{}s it. No library, no subproject, no dependencies, no complex build system. The class is written in vanilla C++11. All in all, everything should require no adjustment of your compiler flags or project settings.
\item {\bfseries Serious testing}. Our class is heavily \href{https://github.com/nlohmann/json/blob/master/test/src/unit.cpp}{\tt unit-\/tested} and covers \href{https://coveralls.io/r/nlohmann/json}{\tt 100\%} of the code, including all exceptional behavior. Furthermore, we checked with \href{http://valgrind.org}{\tt Valgrind} that there are no memory leaks. To maintain high quality, the project is following the \href{https://bestpractices.coreinfrastructure.org/projects/289}{\tt Core Infrastructure Initiative (C\+II) best practices}.
\end{DoxyItemize}

Other aspects were not so important to us\+:


\begin{DoxyItemize}
\item {\bfseries Memory efficiency}. Each J\+S\+ON object has an overhead of one pointer (the maximal size of a union) and one enumeration element (1 byte). The default generalization uses the following C++ data types\+: {\ttfamily std\+::string} for strings, {\ttfamily int64\+\_\+t}, {\ttfamily uint64\+\_\+t} or {\ttfamily double} for numbers, {\ttfamily std\+::map} for objects, {\ttfamily std\+::vector} for arrays, and {\ttfamily bool} for Booleans. However, you can template the generalized class {\ttfamily basic\+\_\+json} to your needs.
\item {\bfseries Speed}. There are certainly \href{https://github.com/miloyip/nativejson-benchmark#parsing-time}{\tt faster J\+S\+ON libraries} out there. However, if your goal is to speed up your development by adding J\+S\+ON support with a single header, then this library is the way to go. If you know how to use a {\ttfamily std\+::vector} or {\ttfamily std\+::map}, you are already set.
\end{DoxyItemize}

See the \href{https://github.com/nlohmann/json/blob/master/.github/CONTRIBUTING.md#please-dont}{\tt contribution guidelines} for more information.

\subsection*{Integration}

The single required source, file {\ttfamily \hyperlink{json_8hpp_source}{json.\+hpp}} is in the {\ttfamily src} directory or \href{https://github.com/nlohmann/json/releases}{\tt released here}. All you need to do is add


\begin{DoxyCode}
\textcolor{preprocessor}{#include "json.hpp"}

\textcolor{comment}{// for convenience}
\textcolor{keyword}{using} \hyperlink{classnlohmann_1_1basic__json}{json} = \hyperlink{namespacenlohmann_a2bfd99e845a2e5cd90aeaf1b1431f474}{nlohmann::json};
\end{DoxyCode}


to the files you want to use J\+S\+ON objects. That\textquotesingle{}s it. Do not forget to set the necessary switches to enable C++11 (e.\+g., {\ttfamily -\/std=c++11} for G\+CC and Clang).

\+:beer\+: If you are using OS X and \href{http://brew.sh}{\tt Homebrew}, just type {\ttfamily brew tap nlohmann/json} and {\ttfamily brew install nlohmann\+\_\+json} and you\textquotesingle{}re set. If you want the bleeding edge rather than the latest release, use {\ttfamily brew install nlohmann\+\_\+json -\/-\/\+H\+E\+AD}.

If you are using the \href{http://mesonbuild.com}{\tt Meson Build System}, then you can wrap this repo as a subproject.

If you are using \href{https://www.conan.io/}{\tt Conan} to manage your dependencies, merely add {\ttfamily jsonformoderncpp/x.\+y.\+z@vthiery/stable} to your {\ttfamily conanfile.\+py}\textquotesingle{}s requires, where {\ttfamily x.\+y.\+z} is the release version you want to use. Please file issues \href{https://github.com/vthiery/conan-jsonformoderncpp/issues}{\tt here} if you experience problems with the packages.

If you are using \href{https://github.com/ruslo/hunter/}{\tt hunter} on your project for external dependencies, then you can use the \href{https://github.com/ruslo/hunter/wiki/pkg.nlohmann_json}{\tt nlohmann\+\_\+json package}. Please see the hunter project for any issues regarding the packaging.

If you are using \href{https://github.com/Microsoft/vcpkg/}{\tt vcpkg} on your project for external dependencies, then you can use the \href{https://github.com/Microsoft/vcpkg/tree/master/ports/nlohmann-json}{\tt nlohmann-\/json package}. Please see the vcpkg project for any issues regarding the packaging.

\+:warning\+: \href{https://github.com/nlohmann/json/wiki/Road-toward-3.0.0}{\tt Version 3.\+0.\+0} is currently under development. Branch {\ttfamily develop} is used for the ongoing work and is probably {\bfseries unstable}. Please use the {\ttfamily master} branch for the last stable version 2.\+1.\+1.

\subsection*{Examples}

Beside the examples below, you may want to check the \href{https://nlohmann.github.io/json/}{\tt documentation} where each function contains a separate code example (e.\+g., check out \href{https://nlohmann.github.io/json/classnlohmann_1_1basic__json_a602f275f0359ab181221384989810604.html#a602f275f0359ab181221384989810604}{\tt {\ttfamily emplace()}}). All \href{https://github.com/nlohmann/json/tree/develop/doc/examples}{\tt example files} can be compiled and executed on their own (e.\+g., file \href{https://github.com/nlohmann/json/blob/develop/doc/examples/emplace.cpp}{\tt emplace.\+cpp}).

\subsubsection*{J\+S\+ON as first-\/class data type}

Here are some examples to give you an idea how to use the class.

Assume you want to create the J\+S\+ON object


\begin{DoxyCode}
\{
  "pi": 3.141,
  "happy": true,
  "name": "Niels",
  "nothing": null,
  "answer": \{
    "everything": 42
  \},
  "list": [1, 0, 2],
  "object": \{
    "currency": "USD",
    "value": 42.99
  \}
\}
\end{DoxyCode}


With the J\+S\+ON class, you could write\+:


\begin{DoxyCode}
\textcolor{comment}{// create an empty structure (null)}
\hyperlink{classnlohmann_1_1basic__json}{json} j;

\textcolor{comment}{// add a number that is stored as double (note the implicit conversion of j to an object)}
j[\textcolor{stringliteral}{"pi"}] = 3.141;

\textcolor{comment}{// add a Boolean that is stored as bool}
j[\textcolor{stringliteral}{"happy"}] = \textcolor{keyword}{true};

\textcolor{comment}{// add a string that is stored as std::string}
j[\textcolor{stringliteral}{"name"}] = \textcolor{stringliteral}{"Niels"};

\textcolor{comment}{// add another null object by passing nullptr}
j[\textcolor{stringliteral}{"nothing"}] = \textcolor{keyword}{nullptr};

\textcolor{comment}{// add an object inside the object}
j[\textcolor{stringliteral}{"answer"}][\textcolor{stringliteral}{"everything"}] = 42;

\textcolor{comment}{// add an array that is stored as std::vector (using an initializer list)}
j[\textcolor{stringliteral}{"list"}] = \{ 1, 0, 2 \};

\textcolor{comment}{// add another object (using an initializer list of pairs)}
j[\textcolor{stringliteral}{"object"}] = \{ \{\textcolor{stringliteral}{"currency"}, \textcolor{stringliteral}{"USD"}\}, \{\textcolor{stringliteral}{"value"}, 42.99\} \};

\textcolor{comment}{// instead, you could also write (which looks very similar to the JSON above)}
\hyperlink{classnlohmann_1_1basic__json}{json} j2 = \{
  \{\textcolor{stringliteral}{"pi"}, 3.141\},
  \{\textcolor{stringliteral}{"happy"}, \textcolor{keyword}{true}\},
  \{\textcolor{stringliteral}{"name"}, \textcolor{stringliteral}{"Niels"}\},
  \{\textcolor{stringliteral}{"nothing"}, \textcolor{keyword}{nullptr}\},
  \{\textcolor{stringliteral}{"answer"}, \{
    \{\textcolor{stringliteral}{"everything"}, 42\}
  \}\},
  \{\textcolor{stringliteral}{"list"}, \{1, 0, 2\}\},
  \{\textcolor{stringliteral}{"object"}, \{
    \{\textcolor{stringliteral}{"currency"}, \textcolor{stringliteral}{"USD"}\},
    \{\textcolor{stringliteral}{"value"}, 42.99\}
  \}\}
\};
\end{DoxyCode}


Note that in all these cases, you never need to \char`\"{}tell\char`\"{} the compiler which J\+S\+ON value you want to use. If you want to be explicit or express some edge cases, the functions {\ttfamily \hyperlink{classnlohmann_1_1basic__json_aa80485befaffcadaa39965494e0b4d2e}{json\+::array}} and {\ttfamily \hyperlink{classnlohmann_1_1basic__json_aa13f7c0615867542ce80337cbcf13ada}{json\+::object}} will help\+:


\begin{DoxyCode}
\textcolor{comment}{// a way to express the empty array []}
\hyperlink{classnlohmann_1_1basic__json}{json} empty\_array\_explicit = \hyperlink{classnlohmann_1_1basic__json_aa80485befaffcadaa39965494e0b4d2e}{json::array}();

\textcolor{comment}{// ways to express the empty object \{\}}
\hyperlink{classnlohmann_1_1basic__json}{json} empty\_object\_implicit = \hyperlink{classnlohmann_1_1basic__json}{json}(\{\});
\hyperlink{classnlohmann_1_1basic__json}{json} empty\_object\_explicit = \hyperlink{classnlohmann_1_1basic__json_aa13f7c0615867542ce80337cbcf13ada}{json::object}();

\textcolor{comment}{// a way to express an \_array\_ of key/value pairs [["currency", "USD"], ["value", 42.99]]}
\hyperlink{classnlohmann_1_1basic__json}{json} array\_not\_object = \{ \hyperlink{classnlohmann_1_1basic__json_aa80485befaffcadaa39965494e0b4d2e}{json::array}(\{\textcolor{stringliteral}{"currency"}, \textcolor{stringliteral}{"USD"}\}), 
      \hyperlink{classnlohmann_1_1basic__json_aa80485befaffcadaa39965494e0b4d2e}{json::array}(\{\textcolor{stringliteral}{"value"}, 42.99\}) \};
\end{DoxyCode}


\subsubsection*{Serialization / Deserialization}

\paragraph*{To/from strings}

You can create an object (deserialization) by appending {\ttfamily \+\_\+json} to a string literal\+:


\begin{DoxyCode}
\textcolor{comment}{// create object from string literal}
\hyperlink{classnlohmann_1_1basic__json}{json} j = \textcolor{stringliteral}{"\{ \(\backslash\)"happy\(\backslash\)": true, \(\backslash\)"pi\(\backslash\)": 3.141 \}"}\_json;

\textcolor{comment}{// or even nicer with a raw string literal}
\textcolor{keyword}{auto} j2 = R\textcolor{stringliteral}{"(}
\textcolor{stringliteral}{  \{}
\textcolor{stringliteral}{    "happy": true,}
\textcolor{stringliteral}{    "pi": 3.141}
\textcolor{stringliteral}{  \}}
\textcolor{stringliteral}{)"\_json;}
\end{DoxyCode}


Note that without appending the {\ttfamily \+\_\+json} suffix, the passed string literal is not parsed, but just used as J\+S\+ON string value. That is, {\ttfamily json j = \char`\"{}\{ \textbackslash{}\char`\"{}happy"\+: true, "pi"\+: 3.\+141 \}\char`\"{}$<$/tt$>$ would just store the string $<$tt$>$\char`\"{}\{ \char`\"{}happy\char`\"{}\+: true, \char`\"{}pi\char`\"{}\+: 3.\+141 \}"} rather than parsing the actual object.

The above example can also be expressed explicitly using {\ttfamily \hyperlink{classnlohmann_1_1basic__json_aa9676414f2e36383c4b181fe856aa3c0}{json\+::parse()}}\+:


\begin{DoxyCode}
\textcolor{comment}{// parse explicitly}
\textcolor{keyword}{auto} j3 = \hyperlink{classnlohmann_1_1basic__json_aa9676414f2e36383c4b181fe856aa3c0}{json::parse}(\textcolor{stringliteral}{"\{ \(\backslash\)"happy\(\backslash\)": true, \(\backslash\)"pi\(\backslash\)": 3.141 \}"});
\end{DoxyCode}


You can also get a string representation (serialize)\+:


\begin{DoxyCode}
\textcolor{comment}{// explicit conversion to string}
std::string s = j.\hyperlink{classnlohmann_1_1basic__json_a5adea76fedba9898d404fef8598aa663}{dump}();    \textcolor{comment}{// \{\(\backslash\)"happy\(\backslash\)":true,\(\backslash\)"pi\(\backslash\)":3.141\}}

\textcolor{comment}{// serialization with pretty printing}
\textcolor{comment}{// pass in the amount of spaces to indent}
std::cout << j.\hyperlink{classnlohmann_1_1basic__json_a5adea76fedba9898d404fef8598aa663}{dump}(4) << std::endl;
\textcolor{comment}{// \{}
\textcolor{comment}{//     "happy": true,}
\textcolor{comment}{//     "pi": 3.141}
\textcolor{comment}{// \}}
\end{DoxyCode}


\paragraph*{To/from streams (e.\+g. files, string streams)}

You can also use streams to serialize and deserialize\+:


\begin{DoxyCode}
\textcolor{comment}{// deserialize from standard input}
\hyperlink{classnlohmann_1_1basic__json}{json} j;
std::cin >> j;

\textcolor{comment}{// serialize to standard output}
std::cout << j;

\textcolor{comment}{// the setw manipulator was overloaded to set the indentation for pretty printing}
std::cout << std::setw(4) << j << std::endl;
\end{DoxyCode}


These operators work for any subclasses of {\ttfamily std\+::istream} or {\ttfamily std\+::ostream}. Here is the same example with files\+:


\begin{DoxyCode}
\textcolor{comment}{// read a JSON file}
std::ifstream i(\textcolor{stringliteral}{"file.json"});
\hyperlink{classnlohmann_1_1basic__json}{json} j;
i >> j;

\textcolor{comment}{// write prettified JSON to another file}
std::ofstream o(\textcolor{stringliteral}{"pretty.json"});
o << std::setw(4) << j << std::endl;
\end{DoxyCode}


Please note that setting the exception bit for {\ttfamily failbit} is inappropriate for this use case. It will result in program termination due to the {\ttfamily noexcept} specifier in use.

\paragraph*{Read from iterator range}

You can also read J\+S\+ON from an iterator range; that is, from any container accessible by iterators whose content is stored as contiguous byte sequence, for instance a {\ttfamily std\+::vector$<$std\+::uint8\+\_\+t$>$}\+:


\begin{DoxyCode}
std::vector<std::uint8\_t> v = \{\textcolor{charliteral}{'t'}, \textcolor{charliteral}{'r'}, \textcolor{charliteral}{'u'}, \textcolor{charliteral}{'e'}\};
\hyperlink{classnlohmann_1_1basic__json}{json} j = \hyperlink{classnlohmann_1_1basic__json_aa9676414f2e36383c4b181fe856aa3c0}{json::parse}(v.begin(), v.end());
\end{DoxyCode}


You may leave the iterators for the range \mbox{[}begin, end)\+:


\begin{DoxyCode}
std::vector<std::uint8\_t> v = \{\textcolor{charliteral}{'t'}, \textcolor{charliteral}{'r'}, \textcolor{charliteral}{'u'}, \textcolor{charliteral}{'e'}\};
\hyperlink{classnlohmann_1_1basic__json}{json} j = \hyperlink{classnlohmann_1_1basic__json_aa9676414f2e36383c4b181fe856aa3c0}{json::parse}(v);
\end{DoxyCode}


\subsubsection*{S\+T\+L-\/like access}

We designed the J\+S\+ON class to behave just like an S\+TL container. In fact, it satisfies the \href{http://en.cppreference.com/w/cpp/concept/ReversibleContainer}{\tt {\bfseries Reversible\+Container}} requirement.


\begin{DoxyCode}
\textcolor{comment}{// create an array using push\_back}
\hyperlink{classnlohmann_1_1basic__json}{json} j;
j.\hyperlink{classnlohmann_1_1basic__json_ac8e523ddc8c2dd7e5d2daf0d49a9c0d7}{push\_back}(\textcolor{stringliteral}{"foo"});
j.\hyperlink{classnlohmann_1_1basic__json_ac8e523ddc8c2dd7e5d2daf0d49a9c0d7}{push\_back}(1);
j.\hyperlink{classnlohmann_1_1basic__json_ac8e523ddc8c2dd7e5d2daf0d49a9c0d7}{push\_back}(\textcolor{keyword}{true});

\textcolor{comment}{// also use emplace\_back}
j.\hyperlink{classnlohmann_1_1basic__json_aacf5eed15a8b66fb1e88910707a5e229}{emplace\_back}(1.78);

\textcolor{comment}{// iterate the array}
\textcolor{keywordflow}{for} (\hyperlink{classnlohmann_1_1detail_1_1iter__impl}{json::iterator} it = j.\hyperlink{classnlohmann_1_1basic__json_a0ff28dac23f2bdecee9564d07f51dcdc}{begin}(); it != j.\hyperlink{classnlohmann_1_1basic__json_a13e032a02a7fd8a93fdddc2fcbc4763c}{end}(); ++it) \{
  std::cout << *it << \textcolor{charliteral}{'\(\backslash\)n'};
\}

\textcolor{comment}{// range-based for}
\textcolor{keywordflow}{for} (\textcolor{keyword}{auto}& element : j) \{
  std::cout << element << \textcolor{charliteral}{'\(\backslash\)n'};
\}

\textcolor{comment}{// getter/setter}
\textcolor{keyword}{const} std::string tmp = j[0];
j[1] = 42;
\textcolor{keywordtype}{bool} foo = j.at(2);

\textcolor{comment}{// comparison}
j == \textcolor{stringliteral}{"[\(\backslash\)"foo\(\backslash\)", 1, true]"}\_json;  \textcolor{comment}{// true}

\textcolor{comment}{// other stuff}
j.size();     \textcolor{comment}{// 3 entries}
j.empty();    \textcolor{comment}{// false}
j.type();     \textcolor{comment}{// json::value\_t::array}
j.clear();    \textcolor{comment}{// the array is empty again}

\textcolor{comment}{// convenience type checkers}
j.is\_null();
j.is\_boolean();
j.is\_number();
j.is\_object();
j.is\_array();
j.is\_string();

\textcolor{comment}{// create an object}
\hyperlink{classnlohmann_1_1basic__json}{json} o;
o[\textcolor{stringliteral}{"foo"}] = 23;
o[\textcolor{stringliteral}{"bar"}] = \textcolor{keyword}{false};
o[\textcolor{stringliteral}{"baz"}] = 3.141;

\textcolor{comment}{// also use emplace}
o.\hyperlink{classnlohmann_1_1basic__json_a5338e282d1d02bed389d852dd670d98d}{emplace}(\textcolor{stringliteral}{"weather"}, \textcolor{stringliteral}{"sunny"});

\textcolor{comment}{// special iterator member functions for objects}
\textcolor{keywordflow}{for} (\hyperlink{classnlohmann_1_1detail_1_1iter__impl}{json::iterator} it = o.\hyperlink{classnlohmann_1_1basic__json_a0ff28dac23f2bdecee9564d07f51dcdc}{begin}(); it != o.\hyperlink{classnlohmann_1_1basic__json_a13e032a02a7fd8a93fdddc2fcbc4763c}{end}(); ++it) \{
  std::cout << it.key() << \textcolor{stringliteral}{" : "} << it.value() << \textcolor{stringliteral}{"\(\backslash\)n"};
\}

\textcolor{comment}{// find an entry}
\textcolor{keywordflow}{if} (o.\hyperlink{classnlohmann_1_1basic__json_a89eb3928f57903677051c80534be9cb1}{find}(\textcolor{stringliteral}{"foo"}) != o.\hyperlink{classnlohmann_1_1basic__json_a13e032a02a7fd8a93fdddc2fcbc4763c}{end}()) \{
  \textcolor{comment}{// there is an entry with key "foo"}
\}

\textcolor{comment}{// or simpler using count()}
\textcolor{keywordtype}{int} foo\_present = o.\hyperlink{classnlohmann_1_1basic__json_a0d74bfcf65662f1d66d14c34b0027098}{count}(\textcolor{stringliteral}{"foo"}); \textcolor{comment}{// 1}
\textcolor{keywordtype}{int} fob\_present = o.\hyperlink{classnlohmann_1_1basic__json_a0d74bfcf65662f1d66d14c34b0027098}{count}(\textcolor{stringliteral}{"fob"}); \textcolor{comment}{// 0}

\textcolor{comment}{// delete an entry}
o.\hyperlink{classnlohmann_1_1basic__json_a068a16e76be178e83da6a192916923ed}{erase}(\textcolor{stringliteral}{"foo"});
\end{DoxyCode}


\subsubsection*{Conversion from S\+TL containers}

Any sequence container ({\ttfamily std\+::array}, {\ttfamily std\+::vector}, {\ttfamily std\+::deque}, {\ttfamily std\+::forward\+\_\+list}, {\ttfamily std\+::list}) whose values can be used to construct J\+S\+ON types (e.\+g., integers, floating point numbers, Booleans, string types, or again S\+TL containers described in this section) can be used to create a J\+S\+ON array. The same holds for similar associative containers ({\ttfamily std\+::set}, {\ttfamily std\+::multiset}, {\ttfamily std\+::unordered\+\_\+set}, {\ttfamily std\+::unordered\+\_\+multiset}), but in these cases the order of the elements of the array depends on how the elements are ordered in the respective S\+TL container.


\begin{DoxyCode}
std::vector<int> c\_vector \{1, 2, 3, 4\};
\hyperlink{classnlohmann_1_1basic__json}{json} j\_vec(c\_vector);
\textcolor{comment}{// [1, 2, 3, 4]}

std::deque<double> c\_deque \{1.2, 2.3, 3.4, 5.6\};
\hyperlink{classnlohmann_1_1basic__json}{json} j\_deque(c\_deque);
\textcolor{comment}{// [1.2, 2.3, 3.4, 5.6]}

std::list<bool> c\_list \{\textcolor{keyword}{true}, \textcolor{keyword}{true}, \textcolor{keyword}{false}, \textcolor{keyword}{true}\};
\hyperlink{classnlohmann_1_1basic__json}{json} j\_list(c\_list);
\textcolor{comment}{// [true, true, false, true]}

std::forward\_list<int64\_t> c\_flist \{12345678909876, 23456789098765, 34567890987654, 45678909876543\};
\hyperlink{classnlohmann_1_1basic__json}{json} j\_flist(c\_flist);
\textcolor{comment}{// [12345678909876, 23456789098765, 34567890987654, 45678909876543]}

std::array<unsigned long, 4> c\_array \{\{1, 2, 3, 4\}\};
\hyperlink{classnlohmann_1_1basic__json}{json} j\_array(c\_array);
\textcolor{comment}{// [1, 2, 3, 4]}

std::set<std::string> c\_set \{\textcolor{stringliteral}{"one"}, \textcolor{stringliteral}{"two"}, \textcolor{stringliteral}{"three"}, \textcolor{stringliteral}{"four"}, \textcolor{stringliteral}{"one"}\};
\hyperlink{classnlohmann_1_1basic__json}{json} j\_set(c\_set); \textcolor{comment}{// only one entry for "one" is used}
\textcolor{comment}{// ["four", "one", "three", "two"]}

std::unordered\_set<std::string> c\_uset \{\textcolor{stringliteral}{"one"}, \textcolor{stringliteral}{"two"}, \textcolor{stringliteral}{"three"}, \textcolor{stringliteral}{"four"}, \textcolor{stringliteral}{"one"}\};
\hyperlink{classnlohmann_1_1basic__json}{json} j\_uset(c\_uset); \textcolor{comment}{// only one entry for "one" is used}
\textcolor{comment}{// maybe ["two", "three", "four", "one"]}

std::multiset<std::string> c\_mset \{\textcolor{stringliteral}{"one"}, \textcolor{stringliteral}{"two"}, \textcolor{stringliteral}{"one"}, \textcolor{stringliteral}{"four"}\};
\hyperlink{classnlohmann_1_1basic__json}{json} j\_mset(c\_mset); \textcolor{comment}{// both entries for "one" are used}
\textcolor{comment}{// maybe ["one", "two", "one", "four"]}

std::unordered\_multiset<std::string> c\_umset \{\textcolor{stringliteral}{"one"}, \textcolor{stringliteral}{"two"}, \textcolor{stringliteral}{"one"}, \textcolor{stringliteral}{"four"}\};
\hyperlink{classnlohmann_1_1basic__json}{json} j\_umset(c\_umset); \textcolor{comment}{// both entries for "one" are used}
\textcolor{comment}{// maybe ["one", "two", "one", "four"]}
\end{DoxyCode}


Likewise, any associative key-\/value containers ({\ttfamily std\+::map}, {\ttfamily std\+::multimap}, {\ttfamily std\+::unordered\+\_\+map}, {\ttfamily std\+::unordered\+\_\+multimap}) whose keys can construct an {\ttfamily std\+::string} and whose values can be used to construct J\+S\+ON types (see examples above) can be used to create a J\+S\+ON object. Note that in case of multimaps only one key is used in the J\+S\+ON object and the value depends on the internal order of the S\+TL container.


\begin{DoxyCode}
std::map<std::string, int> c\_map \{ \{\textcolor{stringliteral}{"one"}, 1\}, \{\textcolor{stringliteral}{"two"}, 2\}, \{\textcolor{stringliteral}{"three"}, 3\} \};
\hyperlink{classnlohmann_1_1basic__json}{json} j\_map(c\_map);
\textcolor{comment}{// \{"one": 1, "three": 3, "two": 2 \}}

std::unordered\_map<const char*, double> c\_umap \{ \{\textcolor{stringliteral}{"one"}, 1.2\}, \{\textcolor{stringliteral}{"two"}, 2.3\}, \{\textcolor{stringliteral}{"three"}, 3.4\} \};
\hyperlink{classnlohmann_1_1basic__json}{json} j\_umap(c\_umap);
\textcolor{comment}{// \{"one": 1.2, "two": 2.3, "three": 3.4\}}

std::multimap<std::string, bool> c\_mmap \{ \{\textcolor{stringliteral}{"one"}, \textcolor{keyword}{true}\}, \{\textcolor{stringliteral}{"two"}, \textcolor{keyword}{true}\}, \{\textcolor{stringliteral}{"three"}, \textcolor{keyword}{false}\}, \{\textcolor{stringliteral}{"three"}, \textcolor{keyword}{true}\} \}
      ;
\hyperlink{classnlohmann_1_1basic__json}{json} j\_mmap(c\_mmap); \textcolor{comment}{// only one entry for key "three" is used}
\textcolor{comment}{// maybe \{"one": true, "two": true, "three": true\}}

std::unordered\_multimap<std::string, bool> c\_ummap \{ \{\textcolor{stringliteral}{"one"}, \textcolor{keyword}{true}\}, \{\textcolor{stringliteral}{"two"}, \textcolor{keyword}{true}\}, \{\textcolor{stringliteral}{"three"}, \textcolor{keyword}{false}\}, \{\textcolor{stringliteral}{"
      three"}, \textcolor{keyword}{true}\} \};
\hyperlink{classnlohmann_1_1basic__json}{json} j\_ummap(c\_ummap); \textcolor{comment}{// only one entry for key "three" is used}
\textcolor{comment}{// maybe \{"one": true, "two": true, "three": true\}}
\end{DoxyCode}


\subsubsection*{J\+S\+ON Pointer and J\+S\+ON Patch}

The library supports {\bfseries J\+S\+ON Pointer} (\href{https://tools.ietf.org/html/rfc6901}{\tt R\+FC 6901}) as alternative means to address structured values. On top of this, {\bfseries J\+S\+ON Patch} (\href{https://tools.ietf.org/html/rfc6902}{\tt R\+FC 6902}) allows to describe differences between two J\+S\+ON values -\/ effectively allowing patch and diff operations known from Unix.


\begin{DoxyCode}
\textcolor{comment}{// a JSON value}
\hyperlink{classnlohmann_1_1basic__json}{json} j\_original = R\textcolor{stringliteral}{"(\{}
\textcolor{stringliteral}{  "baz": ["one", "two", "three"],}
\textcolor{stringliteral}{  "foo": "bar"}
\textcolor{stringliteral}{\})"\_json;}
\textcolor{stringliteral}{}
\textcolor{stringliteral}{}\textcolor{comment}{// access members with a JSON pointer (RFC 6901)}
j\_original[\textcolor{stringliteral}{"/baz/1"}\_json\_pointer];
\textcolor{comment}{// "two"}

\textcolor{comment}{// a JSON patch (RFC 6902)}
\hyperlink{classnlohmann_1_1basic__json}{json} j\_patch = R\textcolor{stringliteral}{"([}
\textcolor{stringliteral}{  \{ "op": "replace", "path": "/baz", "value": "boo" \},}
\textcolor{stringliteral}{  \{ "op": "add", "path": "/hello", "value": ["world"] \},}
\textcolor{stringliteral}{  \{ "op": "remove", "path": "/foo"\}}
\textcolor{stringliteral}{])"\_json;}
\textcolor{stringliteral}{}
\textcolor{stringliteral}{}\textcolor{comment}{// apply the patch}
\hyperlink{classnlohmann_1_1basic__json}{json} j\_result = j\_original.\hyperlink{classnlohmann_1_1basic__json_a81e0c41a4a9dff4df2f6973f7f8b2a83}{patch}(j\_patch);
\textcolor{comment}{// \{}
\textcolor{comment}{//    "baz": "boo",}
\textcolor{comment}{//    "hello": ["world"]}
\textcolor{comment}{// \}}

\textcolor{comment}{// calculate a JSON patch from two JSON values}
\hyperlink{classnlohmann_1_1basic__json_a543bd5f7490de54c875b2c0912dc9a49}{json::diff}(j\_result, j\_original);
\textcolor{comment}{// [}
\textcolor{comment}{//   \{ "op":" replace", "path": "/baz", "value": ["one", "two", "three"] \},}
\textcolor{comment}{//   \{ "op": "remove","path": "/hello" \},}
\textcolor{comment}{//   \{ "op": "add", "path": "/foo", "value": "bar" \}}
\textcolor{comment}{// ]}
\end{DoxyCode}


\subsubsection*{Implicit conversions}

The type of the J\+S\+ON object is determined automatically by the expression to store. Likewise, the stored value is implicitly converted.


\begin{DoxyCode}
\textcolor{comment}{// strings}
std::string s1 = \textcolor{stringliteral}{"Hello, world!"};
\hyperlink{classnlohmann_1_1basic__json}{json} js = s1;
std::string s2 = js;

\textcolor{comment}{// Booleans}
\textcolor{keywordtype}{bool} b1 = \textcolor{keyword}{true};
\hyperlink{classnlohmann_1_1basic__json}{json} jb = b1;
\textcolor{keywordtype}{bool} b2 = jb;

\textcolor{comment}{// numbers}
\textcolor{keywordtype}{int} i = 42;
\hyperlink{classnlohmann_1_1basic__json}{json} jn = i;
\textcolor{keywordtype}{double} f = jn;

\textcolor{comment}{// etc.}
\end{DoxyCode}


You can also explicitly ask for the value\+:


\begin{DoxyCode}
std::string vs = js.\hyperlink{classnlohmann_1_1basic__json_a6b187a22994c12c8cae0dd5ee99dc85e}{get}<std::string>();
\textcolor{keywordtype}{bool} vb = jb.\hyperlink{classnlohmann_1_1basic__json_a6b187a22994c12c8cae0dd5ee99dc85e}{get}<\textcolor{keywordtype}{bool}>();
\textcolor{keywordtype}{int} vi = jn.\hyperlink{classnlohmann_1_1basic__json_a6b187a22994c12c8cae0dd5ee99dc85e}{get}<\textcolor{keywordtype}{int}>();

\textcolor{comment}{// etc.}
\end{DoxyCode}


\subsubsection*{Arbitrary types conversions}

Every type can be serialized in J\+S\+ON, not just S\+T\+L-\/containers and scalar types. Usually, you would do something along those lines\+:


\begin{DoxyCode}
\textcolor{keyword}{namespace }ns \{
    \textcolor{comment}{// a simple struct to model a person}
    \textcolor{keyword}{struct }person \{
        std::string name;
        std::string address;
        \textcolor{keywordtype}{int} age;
    \};
\}

ns::person p = \{\textcolor{stringliteral}{"Ned Flanders"}, \textcolor{stringliteral}{"744 Evergreen Terrace"}, 60\};

\textcolor{comment}{// convert to JSON: copy each value into the JSON object}
\hyperlink{classnlohmann_1_1basic__json}{json} j;
j[\textcolor{stringliteral}{"name"}] = p.name;
j[\textcolor{stringliteral}{"address"}] = p.address;
j[\textcolor{stringliteral}{"age"}] = p.age;

\textcolor{comment}{// ...}

\textcolor{comment}{// convert from JSON: copy each value from the JSON object}
ns::person p \{
    j[\textcolor{stringliteral}{"name"}].\hyperlink{classnlohmann_1_1basic__json_a6b187a22994c12c8cae0dd5ee99dc85e}{get}<std::string>(),
    j[\textcolor{stringliteral}{"address"}].get<std::string>(),
    j[\textcolor{stringliteral}{"age"}].\hyperlink{classnlohmann_1_1basic__json_a6b187a22994c12c8cae0dd5ee99dc85e}{get}<\textcolor{keywordtype}{int}>()
\};
\end{DoxyCode}


It works, but that\textquotesingle{}s quite a lot of boilerplate... Fortunately, there\textquotesingle{}s a better way\+:


\begin{DoxyCode}
\textcolor{comment}{// create a person}
ns::person p \{\textcolor{stringliteral}{"Ned Flanders"}, \textcolor{stringliteral}{"744 Evergreen Terrace"}, 60\};

\textcolor{comment}{// conversion: person -> json}
\hyperlink{classnlohmann_1_1basic__json}{json} j = p;

std::cout << j << std::endl;
\textcolor{comment}{// \{"address":"744 Evergreen Terrace","age":60,"name":"Ned Flanders"\}}

\textcolor{comment}{// conversion: json -> person}
ns::person p2 = j;

\textcolor{comment}{// that's it}
assert(p == p2);
\end{DoxyCode}


\paragraph*{Basic usage}

To make this work with one of your types, you only need to provide two functions\+:


\begin{DoxyCode}
\textcolor{keyword}{using} \hyperlink{namespacenlohmann_a2bfd99e845a2e5cd90aeaf1b1431f474}{nlohmann::json};

\textcolor{keyword}{namespace }ns \{
    \textcolor{keywordtype}{void} to\_json(\hyperlink{classnlohmann_1_1basic__json}{json}& j, \textcolor{keyword}{const} person& p) \{
        j = \hyperlink{classnlohmann_1_1basic__json}{json}\{\{\textcolor{stringliteral}{"name"}, p.name\}, \{\textcolor{stringliteral}{"address"}, p.address\}, \{\textcolor{stringliteral}{"age"}, p.age\}\};
    \}

    \textcolor{keywordtype}{void} from\_json(\textcolor{keyword}{const} \hyperlink{classnlohmann_1_1basic__json}{json}& j, person& p) \{
        p.name = j.\hyperlink{classnlohmann_1_1basic__json_a73ae333487310e3302135189ce8ff5d8}{at}(\textcolor{stringliteral}{"name"}).\hyperlink{classnlohmann_1_1basic__json_a6b187a22994c12c8cae0dd5ee99dc85e}{get}<std::string>();
        p.address = j.\hyperlink{classnlohmann_1_1basic__json_a73ae333487310e3302135189ce8ff5d8}{at}(\textcolor{stringliteral}{"address"}).\hyperlink{classnlohmann_1_1basic__json_a6b187a22994c12c8cae0dd5ee99dc85e}{get}<std::string>();
        p.age = j.\hyperlink{classnlohmann_1_1basic__json_a73ae333487310e3302135189ce8ff5d8}{at}(\textcolor{stringliteral}{"age"}).\hyperlink{classnlohmann_1_1basic__json_a6b187a22994c12c8cae0dd5ee99dc85e}{get}<\textcolor{keywordtype}{int}>();
    \}
\} \textcolor{comment}{// namespace ns}
\end{DoxyCode}


That\textquotesingle{}s all! When calling the {\ttfamily json} constructor with your type, your custom {\ttfamily to\+\_\+json} method will be automatically called. Likewise, when calling {\ttfamily get$<$your\+\_\+type$>$()}, the {\ttfamily from\+\_\+json} method will be called.

Some important things\+:


\begin{DoxyItemize}
\item Those methods {\bfseries M\+U\+ST} be in your type\textquotesingle{}s namespace (which can be the global namespace), or the library will not be able to locate them (in this example, they are in namespace {\ttfamily ns}, where {\ttfamily person} is defined).
\item When using {\ttfamily get$<$your\+\_\+type$>$()}, {\ttfamily your\+\_\+type} {\bfseries M\+U\+ST} be \href{http://en.cppreference.com/w/cpp/concept/DefaultConstructible}{\tt Default\+Constructible}. (There is a way to bypass this requirement described later.)
\item In function {\ttfamily from\+\_\+json}, use function \href{https://nlohmann.github.io/json/classnlohmann_1_1basic__json_a93403e803947b86f4da2d1fb3345cf2c.html#a93403e803947b86f4da2d1fb3345cf2c}{\tt {\ttfamily at()}} to access the object values rather than {\ttfamily operator\mbox{[}\mbox{]}}. In case a key does not exist, {\ttfamily at} throws an exception that you can handle, whereas {\ttfamily operator\mbox{[}\mbox{]}} exhibits undefined behavior.
\item In case your type contains several {\ttfamily operator=} definitions, code like {\ttfamily your\+\_\+variable = your\+\_\+json;} \href{https://github.com/nlohmann/json/issues/667}{\tt may not compile}. You need to write {\ttfamily your\+\_\+variable = your\+\_\+json.\+get$<$decltype your\+\_\+variable$>$();} instead.
\item You do not need to add serializers or deserializers for S\+TL types like {\ttfamily std\+::vector}\+: the library already implements these.
\item Be careful with the definition order of the {\ttfamily from\+\_\+json}/{\ttfamily to\+\_\+json} functions\+: If a type {\ttfamily B} has a member of type {\ttfamily A}, you {\bfseries M\+U\+ST} define {\ttfamily to\+\_\+json(\+A)} before {\ttfamily to\+\_\+json(\+B)}. Look at \href{https://github.com/nlohmann/json/issues/561}{\tt issue 561} for more details.
\end{DoxyItemize}

\paragraph*{How do I convert third-\/party types?}

This requires a bit more advanced technique. But first, let\textquotesingle{}s see how this conversion mechanism works\+:

The library uses {\bfseries J\+S\+ON Serializers} to convert types to json. The default serializer for {\ttfamily \hyperlink{namespacenlohmann_a2bfd99e845a2e5cd90aeaf1b1431f474}{nlohmann\+::json}} is {\ttfamily \hyperlink{structnlohmann_1_1adl__serializer}{nlohmann\+::adl\+\_\+serializer}} (A\+DL means \href{http://en.cppreference.com/w/cpp/language/adl}{\tt Argument-\/\+Dependent Lookup}).

It is implemented like this (simplified)\+:


\begin{DoxyCode}
\textcolor{keyword}{template} <\textcolor{keyword}{typename} T>
\textcolor{keyword}{struct }adl\_serializer \{
    \textcolor{keyword}{static} \textcolor{keywordtype}{void} to\_json(\hyperlink{classnlohmann_1_1basic__json}{json}& j, \textcolor{keyword}{const} T& value) \{
        \textcolor{comment}{// calls the "to\_json" method in T's namespace}
    \}

    \textcolor{keyword}{static} \textcolor{keywordtype}{void} from\_json(\textcolor{keyword}{const} \hyperlink{classnlohmann_1_1basic__json}{json}& j, T& value) \{
        \textcolor{comment}{// same thing, but with the "from\_json" method}
    \}
\};
\end{DoxyCode}


This serializer works fine when you have control over the type\textquotesingle{}s namespace. However, what about {\ttfamily boost\+::optional}, or {\ttfamily std\+::filesystem\+::path} (C++17)? Hijacking the {\ttfamily boost} namespace is pretty bad, and it\textquotesingle{}s illegal to add something other than template specializations to {\ttfamily std}...

To solve this, you need to add a specialization of {\ttfamily adl\+\_\+serializer} to the {\ttfamily nlohmann} namespace, here\textquotesingle{}s an example\+:


\begin{DoxyCode}
\textcolor{comment}{// partial specialization (full specialization works too)}
\textcolor{keyword}{namespace }\hyperlink{namespacenlohmann}{nlohmann} \{
    \textcolor{keyword}{template} <\textcolor{keyword}{typename} T>
    \textcolor{keyword}{struct }adl\_serializer<boost::optional<T>> \{
        \textcolor{keyword}{static} \textcolor{keywordtype}{void} \hyperlink{structnlohmann_1_1adl__serializer_adf8cd96afe6ab243b67392dfe35ace89}{to\_json}(\hyperlink{classnlohmann_1_1basic__json}{json}& j, \textcolor{keyword}{const} boost::optional<T>& opt) \{
            \textcolor{keywordflow}{if} (opt == boost::none) \{
                j = \textcolor{keyword}{nullptr};
            \} \textcolor{keywordflow}{else} \{
              j = *opt; \textcolor{comment}{// this will call adl\_serializer<T>::to\_json which will}
                        \textcolor{comment}{// find the free function to\_json in T's namespace!}
            \}
        \}

        \textcolor{keyword}{static} \textcolor{keywordtype}{void} \hyperlink{structnlohmann_1_1adl__serializer_ab39cad07c1a2bf4414d6cae5215b4e7a}{from\_json}(\textcolor{keyword}{const} \hyperlink{classnlohmann_1_1basic__json}{json}& j, boost::optional<T>& opt) \{
            \textcolor{keywordflow}{if} (j.\hyperlink{classnlohmann_1_1basic__json_a8faa039ca82427ed29c486ffd00600c3}{is\_null}()) \{
                opt = boost::none;
            \} \textcolor{keywordflow}{else} \{
                opt = j.\hyperlink{classnlohmann_1_1basic__json_a6b187a22994c12c8cae0dd5ee99dc85e}{get}<T>(); \textcolor{comment}{// same as above, but with }
                                  \textcolor{comment}{// adl\_serializer<T>::from\_json}
            \}
        \}
    \};
\}
\end{DoxyCode}


\paragraph*{How can I use {\ttfamily get()} for non-\/default constructible/non-\/copyable types?}

There is a way, if your type is \href{http://en.cppreference.com/w/cpp/concept/MoveConstructible}{\tt Move\+Constructible}. You will need to specialize the {\ttfamily adl\+\_\+serializer} as well, but with a special {\ttfamily from\+\_\+json} overload\+:


\begin{DoxyCode}
\textcolor{keyword}{struct }move\_only\_type \{
    move\_only\_type() = \textcolor{keyword}{delete};
    move\_only\_type(\textcolor{keywordtype}{int} ii): i(ii) \{\}
    move\_only\_type(\textcolor{keyword}{const} move\_only\_type&) = \textcolor{keyword}{delete};
    move\_only\_type(move\_only\_type&&) = \textcolor{keywordflow}{default};

    \textcolor{keywordtype}{int} i;
\};

\textcolor{keyword}{namespace }\hyperlink{namespacenlohmann}{nlohmann} \{
    \textcolor{keyword}{template} <>
    \textcolor{keyword}{struct }adl\_serializer<move\_only\_type> \{
        \textcolor{comment}{// note: the return type is no longer 'void', and the method only takes}
        \textcolor{comment}{// one argument}
        \textcolor{keyword}{static} move\_only\_type from\_json(\textcolor{keyword}{const} \hyperlink{classnlohmann_1_1basic__json}{json}& j) \{
            \textcolor{keywordflow}{return} \{j.\hyperlink{classnlohmann_1_1basic__json_a6b187a22994c12c8cae0dd5ee99dc85e}{get}<\textcolor{keywordtype}{int}>()\};
        \}

        \textcolor{comment}{// Here's the catch! You must provide a to\_json method! Otherwise you}
        \textcolor{comment}{// will not be able to convert move\_only\_type to json, since you fully}
        \textcolor{comment}{// specialized adl\_serializer on that type}
        \textcolor{keyword}{static} \textcolor{keywordtype}{void} to\_json(\hyperlink{classnlohmann_1_1basic__json}{json}& j, move\_only\_type t) \{
            j = t.i;
        \}
    \};
\}
\end{DoxyCode}


\paragraph*{Can I write my own serializer? (Advanced use)}

Yes. You might want to take a look at \href{https://github.com/nlohmann/json/blob/develop/test/src/unit-udt.cpp}{\tt {\ttfamily unit-\/udt.\+cpp}} in the test suite, to see a few examples.

If you write your own serializer, you\textquotesingle{}ll need to do a few things\+:


\begin{DoxyItemize}
\item use a different {\ttfamily basic\+\_\+json} alias than {\ttfamily \hyperlink{namespacenlohmann_a2bfd99e845a2e5cd90aeaf1b1431f474}{nlohmann\+::json}} (the last template parameter of {\ttfamily basic\+\_\+json} is the {\ttfamily J\+S\+O\+N\+Serializer})
\item use your {\ttfamily basic\+\_\+json} alias (or a template parameter) in all your {\ttfamily to\+\_\+json}/{\ttfamily from\+\_\+json} methods
\item use {\ttfamily nlohmann\+::to\+\_\+json} and {\ttfamily nlohmann\+::from\+\_\+json} when you need A\+DL
\end{DoxyItemize}

Here is an example, without simplifications, that only accepts types with a size $<$= 32, and uses A\+DL.


\begin{DoxyCode}
\textcolor{comment}{// You should use void as a second template argument}
\textcolor{comment}{// if you don't need compile-time checks on T}
template<typename T, typename SFINAE = typename std::enable\_if<sizeof(T) <= 32>::type>
\textcolor{keyword}{struct }less\_than\_32\_serializer \{
    \textcolor{keyword}{template} <\textcolor{keyword}{typename} BasicJsonType>
    \textcolor{keyword}{static} \textcolor{keywordtype}{void} to\_json(BasicJsonType& j, T value) \{
        \textcolor{comment}{// we want to use ADL, and call the correct to\_json overload}
        \textcolor{keyword}{using} nlohmann::to\_json; \textcolor{comment}{// this method is called by adl\_serializer,}
                                 \textcolor{comment}{// this is where the magic happens}
        to\_json(j, value);
    \}

    \textcolor{keyword}{template} <\textcolor{keyword}{typename} BasicJsonType>
    \textcolor{keyword}{static} \textcolor{keywordtype}{void} from\_json(\textcolor{keyword}{const} BasicJsonType& j, T& value) \{
        \textcolor{comment}{// same thing here}
        \textcolor{keyword}{using} nlohmann::from\_json;
        from\_json(j, value);
    \}
\};
\end{DoxyCode}


Be {\bfseries very} careful when reimplementing your serializer, you can stack overflow if you don\textquotesingle{}t pay attention\+:


\begin{DoxyCode}
\textcolor{keyword}{template} <\textcolor{keyword}{typename} T, \textcolor{keywordtype}{void}>
\textcolor{keyword}{struct }bad\_serializer
\{
    \textcolor{keyword}{template} <\textcolor{keyword}{typename} BasicJsonType>
    \textcolor{keyword}{static} \textcolor{keywordtype}{void} to\_json(BasicJsonType& j, \textcolor{keyword}{const} T& value) \{
      \textcolor{comment}{// this calls BasicJsonType::json\_serializer<T>::to\_json(j, value);}
      \textcolor{comment}{// if BasicJsonType::json\_serializer == bad\_serializer ... oops!}
      j = value;
    \}

    \textcolor{keyword}{template} <\textcolor{keyword}{typename} BasicJsonType>
    \textcolor{keyword}{static} \textcolor{keywordtype}{void} to\_json(\textcolor{keyword}{const} BasicJsonType& j, T& value) \{
      \textcolor{comment}{// this calls BasicJsonType::json\_serializer<T>::from\_json(j, value);}
      \textcolor{comment}{// if BasicJsonType::json\_serializer == bad\_serializer ... oops!}
      value = j.template get<T>(); \textcolor{comment}{// oops!}
    \}
\};
\end{DoxyCode}


\subsubsection*{Binary formats (C\+B\+OR and Message\+Pack)}

Though J\+S\+ON is a ubiquitous data format, it is not a very compact format suitable for data exchange, for instance over a network. Hence, the library supports \href{http://cbor.io}{\tt C\+B\+OR} (Concise Binary Object Representation) and \href{http://msgpack.org}{\tt Message\+Pack} to efficiently encode J\+S\+ON values to byte vectors and to decode such vectors.


\begin{DoxyCode}
\textcolor{comment}{// create a JSON value}
\hyperlink{classnlohmann_1_1basic__json}{json} j = R\textcolor{stringliteral}{"(\{"compact": true, "schema": 0\})"\_json;}
\textcolor{stringliteral}{}
\textcolor{stringliteral}{}\textcolor{comment}{// serialize to CBOR}
std::vector<std::uint8\_t> v\_cbor = \hyperlink{classnlohmann_1_1basic__json_a2566783e190dec524bf3445b322873b8}{json::to\_cbor}(j);

\textcolor{comment}{// 0xa2, 0x67, 0x63, 0x6f, 0x6d, 0x70, 0x61, 0x63, 0x74, 0xf5, 0x66, 0x73, 0x63, 0x68, 0x65, 0x6d, 0x61,
       0x00}

\textcolor{comment}{// roundtrip}
\hyperlink{classnlohmann_1_1basic__json}{json} j\_from\_cbor = \hyperlink{classnlohmann_1_1basic__json_aa9be366b887378bb10c0f1ab510c2f0c}{json::from\_cbor}(v\_cbor);

\textcolor{comment}{// serialize to MessagePack}
std::vector<std::uint8\_t> v\_msgpack = \hyperlink{classnlohmann_1_1basic__json_a09ca1dc273d226afe0ca83a9d7438d9c}{json::to\_msgpack}(j);

\textcolor{comment}{// 0x82, 0xa7, 0x63, 0x6f, 0x6d, 0x70, 0x61, 0x63, 0x74, 0xc3, 0xa6, 0x73, 0x63, 0x68, 0x65, 0x6d, 0x61,
       0x00}

\textcolor{comment}{// roundtrip}
\hyperlink{classnlohmann_1_1basic__json}{json} j\_from\_msgpack = \hyperlink{classnlohmann_1_1basic__json_aab804530006701b136ef9a0bc961434b}{json::from\_msgpack}(v\_msgpack);
\end{DoxyCode}


\subsection*{Supported compilers}

Though it\textquotesingle{}s 2016 already, the support for C++11 is still a bit sparse. Currently, the following compilers are known to work\+:


\begin{DoxyItemize}
\item G\+CC 4.\+9 -\/ 7.\+1 (and possibly later)
\item Clang 3.\+4 -\/ 5.\+0 (and possibly later)
\item Intel C++ Compiler 17.\+0.\+2 (and possibly later)
\item Microsoft Visual C++ 2015 / Build Tools 14.\+0.\+25123.\+0 (and possibly later)
\item Microsoft Visual C++ 2017 / Build Tools 15.\+1.\+548.\+43366 (and possibly later)
\end{DoxyItemize}

I would be happy to learn about other compilers/versions.

Please note\+:


\begin{DoxyItemize}
\item G\+CC 4.\+8 does not work because of two bugs (\href{https://gcc.gnu.org/bugzilla/show_bug.cgi?id=55817}{\tt 55817} and \href{https://gcc.gnu.org/bugzilla/show_bug.cgi?id=57824}{\tt 57824}) in the C++11 support. Note there is a \href{https://github.com/nlohmann/json/pull/212}{\tt pull request} to fix some of the issues.
\item Android defaults to using very old compilers and C++ libraries. To fix this, add the following to your {\ttfamily Application.\+mk}. This will switch to the L\+L\+VM C++ library, the Clang compiler, and enable C++11 and other features disabled by default.

``` A\+P\+P\+\_\+\+S\+TL \+:= c++\+\_\+shared N\+D\+K\+\_\+\+T\+O\+O\+L\+C\+H\+A\+I\+N\+\_\+\+V\+E\+R\+S\+I\+ON \+:= clang3.\+6 A\+P\+P\+\_\+\+C\+P\+P\+F\+L\+A\+GS += -\/frtti -\/fexceptions ```

The code compiles successfully with \href{https://developer.android.com/ndk/index.html?hl=ml}{\tt Android N\+DK}, Revision 9 -\/ 11 (and possibly later) and \href{https://www.crystax.net/en/android/ndk}{\tt CrystaX\textquotesingle{}s Android N\+DK} version 10.
\item For G\+CC running on Min\+GW or Android S\+DK, the error `\textquotesingle{}to\+\_\+string\textquotesingle{} is not a member of \textquotesingle{}std\textquotesingle{}{\ttfamily (or similarly, for}strtod{\ttfamily ) may occur. Note this is not an issue with the code, but rather with the compiler itself. On Android, see above to build with a newer environment. For Min\+GW, please refer to \mbox{[}this site\mbox{]}(\href{http://tehsausage.com/mingw-to-string}{\tt http\+://tehsausage.\+com/mingw-\/to-\/string}) and \mbox{[}this discussion\mbox{]}(\href{https://github.com/nlohmann/json/issues/136}{\tt https\+://github.\+com/nlohmann/json/issues/136}) for information on how to fix this bug. For Android N\+DK using}A\+P\+P\+\_\+\+S\+TL \+:= gnustl\+\_\+static`, please refer to \href{https://github.com/nlohmann/json/issues/219}{\tt this discussion}.
\end{DoxyItemize}

The following compilers are currently used in continuous integration at \href{https://travis-ci.org/nlohmann/json}{\tt Travis} and \href{https://ci.appveyor.com/project/nlohmann/json}{\tt App\+Veyor}\+:

\tabulinesep=1mm
\begin{longtabu} spread 0pt [c]{*{3}{|X[-1]}|}
\hline
\rowcolor{\tableheadbgcolor}\textbf{ Compiler }&\textbf{ Operating System }&\textbf{ Version String  }\\\cline{1-3}
\endfirsthead
\hline
\endfoot
\hline
\rowcolor{\tableheadbgcolor}\textbf{ Compiler }&\textbf{ Operating System }&\textbf{ Version String  }\\\cline{1-3}
\endhead
G\+CC 4.\+9.\+4 &Ubuntu 14.\+04.\+5 L\+TS &g++-\/4.9 (Ubuntu 4.\+9.\+4-\/2ubuntu1$\sim$14.04.\+1) 4.\+9.\+4 \\\cline{1-3}
G\+CC 5.\+4.\+1 &Ubuntu 14.\+04.\+5 L\+TS &g++-\/5 (Ubuntu 5.\+4.\+1-\/2ubuntu1$\sim$14.04) 5.\+4.\+1 20160904 \\\cline{1-3}
G\+CC 6.\+3.\+0 &Ubuntu 14.\+04.\+5 L\+TS &g++-\/6 (Ubuntu/\+Linaro 6.\+3.\+0-\/18ubuntu2$\sim$14.04) 6.\+3.\+0 20170519 \\\cline{1-3}
G\+CC 7.\+1.\+0 &Ubuntu 14.\+04.\+5 L\+TS &g++-\/7 (Ubuntu 7.\+1.\+0-\/5ubuntu2$\sim$14.04) 7.\+1.\+0 \\\cline{1-3}
Clang 3.\+5.\+0 &Ubuntu 14.\+04.\+5 L\+TS &clang version 3.\+5.\+0-\/4ubuntu2$\sim$trusty2 (tags/\+R\+E\+L\+E\+A\+S\+E\+\_\+350/final) \\\cline{1-3}
Clang 3.\+6.\+2 &Ubuntu 14.\+04.\+5 L\+TS &clang version 3.\+6.\+2-\/svn240577-\/1$\sim$exp1 (branches/release\+\_\+36) \\\cline{1-3}
Clang 3.\+7.\+1 &Ubuntu 14.\+04.\+5 L\+TS &clang version 3.\+7.\+1-\/svn253571-\/1$\sim$exp1 (branches/release\+\_\+37) \\\cline{1-3}
Clang 3.\+8.\+0 &Ubuntu 14.\+04.\+5 L\+TS &clang version 3.\+8.\+0-\/2ubuntu3$\sim$trusty5 (tags/\+R\+E\+L\+E\+A\+S\+E\+\_\+380/final) \\\cline{1-3}
Clang 3.\+9.\+1 &Ubuntu 14.\+04.\+5 L\+TS &clang version 3.\+9.\+1-\/4ubuntu3$\sim$14.04.\+2 (tags/\+R\+E\+L\+E\+A\+S\+E\+\_\+391/rc2) \\\cline{1-3}
Clang 4.\+0.\+1 &Ubuntu 14.\+04.\+5 L\+TS &clang version 4.\+0.\+1-\/svn305264-\/1$\sim$exp1 (branches/release\+\_\+40) \\\cline{1-3}
Clang 5.\+0.\+0 &Ubuntu 14.\+04.\+5 L\+TS &clang version 5.\+0.\+0-\/svn310902-\/1$\sim$exp1 (branches/release\+\_\+50) \\\cline{1-3}
Clang Xcode 6.\+4 &Darwin Kernel Version 14.\+3.\+0 (O\+SX 10.\+10.\+3) &Apple L\+L\+VM version 6.\+1.\+0 (clang-\/602.\+0.\+53) (based on L\+L\+VM 3.\+6.\+0svn) \\\cline{1-3}
Clang Xcode 7.\+3 &Darwin Kernel Version 15.\+0.\+0 (O\+SX 10.\+10.\+5) &Apple L\+L\+VM version 7.\+3.\+0 (clang-\/703.\+0.\+29) \\\cline{1-3}
Clang Xcode 8.\+0 &Darwin Kernel Version 15.\+6.\+0 &Apple L\+L\+VM version 8.\+0.\+0 (clang-\/800.\+0.\+38) \\\cline{1-3}
Clang Xcode 8.\+1 &Darwin Kernel Version 16.\+1.\+0 (mac\+OS 10.\+12.\+1) &Apple L\+L\+VM version 8.\+0.\+0 (clang-\/800.\+0.\+42.\+1) \\\cline{1-3}
Clang Xcode 8.\+2 &Darwin Kernel Version 16.\+1.\+0 (mac\+OS 10.\+12.\+1) &Apple L\+L\+VM version 8.\+0.\+0 (clang-\/800.\+0.\+42.\+1) \\\cline{1-3}
Clang Xcode 8.\+3 &Darwin Kernel Version 16.\+5.\+0 (mac\+OS 10.\+12.\+4) &Apple L\+L\+VM version 8.\+1.\+0 (clang-\/802.\+0.\+38) \\\cline{1-3}
Clang Xcode 9 beta &Darwin Kernel Version 16.\+6.\+0 (mac\+OS 10.\+12.\+5) &Apple L\+L\+VM version 9.\+0.\+0 (clang-\/900.\+0.\+26) \\\cline{1-3}
Visual Studio 14 2015 &Windows Server 2012 R2 (x64) &Microsoft (R) Build Engine version 14.\+0.\+25420.\+1 \\\cline{1-3}
Visual Studio 2017 &Windows Server 2016 &Microsoft (R) Build Engine version 15.\+1.\+1012.\+6693 \\\cline{1-3}
\end{longtabu}
\subsection*{License}



The class is licensed under the \href{http://opensource.org/licenses/MIT}{\tt M\+IT License}\+:

Copyright \copyright{} 2013-\/2017 \href{http://nlohmann.me}{\tt Niels Lohmann}

Permission is hereby granted, free of charge, to any person obtaining a copy of this software and associated documentation files (the “\+Software”), to deal in the Software without restriction, including without limitation the rights to use, copy, modify, merge, publish, distribute, sublicense, and/or sell copies of the Software, and to permit persons to whom the Software is furnished to do so, subject to the following conditions\+:

The above copyright notice and this permission notice shall be included in all copies or substantial portions of the Software.

T\+HE S\+O\+F\+T\+W\+A\+RE IS P\+R\+O\+V\+I\+D\+ED “\+AS I\+S”, W\+I\+T\+H\+O\+UT W\+A\+R\+R\+A\+N\+TY OF A\+NY K\+I\+ND, E\+X\+P\+R\+E\+SS OR I\+M\+P\+L\+I\+ED, I\+N\+C\+L\+U\+D\+I\+NG B\+UT N\+OT L\+I\+M\+I\+T\+ED TO T\+HE W\+A\+R\+R\+A\+N\+T\+I\+ES OF M\+E\+R\+C\+H\+A\+N\+T\+A\+B\+I\+L\+I\+TY, F\+I\+T\+N\+E\+SS F\+OR A P\+A\+R\+T\+I\+C\+U\+L\+AR P\+U\+R\+P\+O\+SE A\+ND N\+O\+N\+I\+N\+F\+R\+I\+N\+G\+E\+M\+E\+NT. IN NO E\+V\+E\+NT S\+H\+A\+LL T\+HE A\+U\+T\+H\+O\+RS OR C\+O\+P\+Y\+R\+I\+G\+HT H\+O\+L\+D\+E\+RS BE L\+I\+A\+B\+LE F\+OR A\+NY C\+L\+A\+IM, D\+A\+M\+A\+G\+ES OR O\+T\+H\+ER L\+I\+A\+B\+I\+L\+I\+TY, W\+H\+E\+T\+H\+ER IN AN A\+C\+T\+I\+ON OF C\+O\+N\+T\+R\+A\+CT, T\+O\+RT OR O\+T\+H\+E\+R\+W\+I\+SE, A\+R\+I\+S\+I\+NG F\+R\+OM, O\+UT OF OR IN C\+O\+N\+N\+E\+C\+T\+I\+ON W\+I\+TH T\+HE S\+O\+F\+T\+W\+A\+RE OR T\+HE U\+SE OR O\+T\+H\+ER D\+E\+A\+L\+I\+N\+GS IN T\+HE S\+O\+F\+T\+W\+A\+RE.

\subsection*{Contact}

If you have questions regarding the library, I would like to invite you to \href{https://github.com/nlohmann/json/issues/new}{\tt open an issue at Github}. Please describe your request, problem, or question as detailed as possible, and also mention the version of the library you are using as well as the version of your compiler and operating system. Opening an issue at Github allows other users and contributors to this library to collaborate. For instance, I have little experience with M\+S\+VC, and most issues in this regard have been solved by a growing community. If you have a look at the \href{https://github.com/nlohmann/json/issues?q=is%3Aissue+is%3Aclosed}{\tt closed issues}, you will see that we react quite timely in most cases.

Only if your request would contain confidential information, please \href{mailto:mail@nlohmann.me}{\tt send me an email}.

\subsection*{Thanks}

I deeply appreciate the help of the following people.


\begin{DoxyItemize}
\item \href{https://github.com/Teemperor}{\tt Teemperor} implemented C\+Make support and lcov integration, realized escape and Unicode handling in the string parser, and fixed the J\+S\+ON serialization.
\item \href{https://github.com/elliotgoodrich}{\tt elliotgoodrich} fixed an issue with double deletion in the iterator classes.
\item \href{https://github.com/kirkshoop}{\tt kirkshoop} made the iterators of the class composable to other libraries.
\item \href{https://github.com/wanwc}{\tt wancw} fixed a bug that hindered the class to compile with Clang.
\item Tomas Åblad found a bug in the iterator implementation.
\item \href{https://github.com/jrandall}{\tt Joshua C. Randall} fixed a bug in the floating-\/point serialization.
\item \href{https://github.com/aburgh}{\tt Aaron Burghardt} implemented code to parse streams incrementally. Furthermore, he greatly improved the parser class by allowing the definition of a filter function to discard undesired elements while parsing.
\item \href{https://github.com/dkopecek}{\tt Daniel Kopeček} fixed a bug in the compilation with G\+CC 5.\+0.
\item \href{https://github.com/Florianjw}{\tt Florian Weber} fixed a bug in and improved the performance of the comparison operators.
\item \href{https://github.com/EricMCornelius}{\tt Eric Cornelius} pointed out a bug in the handling with NaN and infinity values. He also improved the performance of the string escaping.
\item \href{https://github.com/likebeta}{\tt 易思龙} implemented a conversion from anonymous enums.
\item \href{https://github.com/kepkin}{\tt kepkin} patiently pushed forward the support for Microsoft Visual studio.
\item \href{https://github.com/gregmarr}{\tt gregmarr} simplified the implementation of reverse iterators and helped with numerous hints and improvements. In particular, he pushed forward the implementation of user-\/defined types.
\item \href{https://github.com/caiovlp}{\tt Caio Luppi} fixed a bug in the Unicode handling.
\item \href{https://github.com/dariomt}{\tt dariomt} fixed some typos in the examples.
\item \href{https://github.com/d-frey}{\tt Daniel Frey} cleaned up some pointers and implemented exception-\/safe memory allocation.
\item \href{https://github.com/ColinH}{\tt Colin Hirsch} took care of a small namespace issue.
\item \href{https://github.com/whoshuu}{\tt Huu Nguyen} correct a variable name in the documentation.
\item \href{https://github.com/silverweed}{\tt Silverweed} overloaded {\ttfamily parse()} to accept an rvalue reference.
\item \href{https://github.com/dariomt}{\tt dariomt} fixed a subtlety in M\+S\+VC type support and implemented the {\ttfamily get\+\_\+ref()} function to get a reference to stored values.
\item \href{https://github.com/ZahlGraf}{\tt Zahl\+Graf} added a workaround that allows compilation using Android N\+DK.
\item \href{https://github.com/whackashoe}{\tt whackashoe} replaced a function that was marked as unsafe by Visual Studio.
\item \href{https://github.com/406345}{\tt 406345} fixed two small warnings.
\item \href{https://github.com/glenfe}{\tt Glen Fernandes} noted a potential portability problem in the {\ttfamily has\+\_\+mapped\+\_\+type} function.
\item \href{https://github.com/nibroc}{\tt Corbin Hughes} fixed some typos in the contribution guidelines.
\item \href{https://github.com/twelsby}{\tt twelsby} fixed the array subscript operator, an issue that failed the M\+S\+VC build, and floating-\/point parsing/dumping. He further added support for unsigned integer numbers and implemented better roundtrip support for parsed numbers.
\item \href{https://github.com/vog}{\tt Volker Diels-\/\+Grabsch} fixed a link in the R\+E\+A\+D\+ME file.
\item \href{https://github.com/msm-}{\tt msm-\/} added support for american fuzzy lop.
\item \href{https://github.com/Annihil}{\tt Annihil} fixed an example in the R\+E\+A\+D\+ME file.
\item \href{https://github.com/Themercee}{\tt Themercee} noted a wrong U\+RL in the R\+E\+A\+D\+ME file.
\item \href{https://github.com/lv-zheng}{\tt Lv Zheng} fixed a namespace issue with {\ttfamily int64\+\_\+t} and {\ttfamily uint64\+\_\+t}.
\item \href{https://github.com/abc100m}{\tt abc100m} analyzed the issues with G\+CC 4.\+8 and proposed a \href{https://github.com/nlohmann/json/pull/212}{\tt partial solution}.
\item \href{https://github.com/zewt}{\tt zewt} added useful notes to the R\+E\+A\+D\+ME file about Android.
\item \href{https://github.com/robertmrk}{\tt Róbert Márki} added a fix to use move iterators and improved the integration via C\+Make.
\item \href{https://github.com/ChrisKitching}{\tt Chris Kitching} cleaned up the C\+Make files.
\item \href{https://github.com/06needhamt}{\tt Tom Needham} fixed a subtle bug with M\+S\+VC 2015 which was also proposed by \href{https://github.com/Epidal}{\tt Michael K.}.
\item \href{https://github.com/thelostt}{\tt Mário Feroldi} fixed a small typo.
\item \href{https://github.com/duncanwerner}{\tt duncanwerner} found a really embarrassing performance regression in the 2.\+0.\+0 release.
\item \href{https://github.com/dtoma}{\tt Damien} fixed one of the last conversion warnings.
\item \href{https://github.com/t-b}{\tt Thomas Braun} fixed a warning in a test case.
\item \href{https://github.com/theodelrieu}{\tt Théo D\+E\+L\+R\+I\+EU} patiently and constructively oversaw the long way toward \href{https://github.com/nlohmann/json/issues/290}{\tt iterator-\/range parsing}. He also implemented the magic behind the serialization/deserialization of user-\/defined types.
\item \href{https://github.com/5tefan}{\tt Stefan} fixed a minor issue in the documentation.
\item \href{https://github.com/vasild}{\tt Vasil Dimov} fixed the documentation regarding conversions from {\ttfamily std\+::multiset}.
\item \href{https://github.com/ChristophJud}{\tt Christoph\+Jud} overworked the C\+Make files to ease project inclusion.
\item \href{https://github.com/vpetrigo}{\tt Vladimir Petrigo} made a S\+F\+I\+N\+AE hack more readable and added Visual Studio 17 to the build matrix.
\item \href{https://github.com/seeekr}{\tt Denis Andrejew} fixed a grammar issue in the R\+E\+A\+D\+ME file.
\item \href{https://github.com/palacaze}{\tt Pierre-\/\+Antoine Lacaze} found a subtle bug in the {\ttfamily dump()} function.
\item \href{https://github.com/TurpentineDistillery}{\tt Turpentine\+Distillery} pointed to \href{http://en.cppreference.com/w/cpp/locale/locale/classic}{\tt {\ttfamily std\+::locale\+::classic()}} to avoid too much locale joggling, found some nice performance improvements in the parser, improved the benchmarking code, and realized locale-\/independent number parsing and printing.
\item \href{https://github.com/cgzones}{\tt cgzones} had an idea how to fix the Coverity scan.
\item \href{https://github.com/jaredgrubb}{\tt Jared Grubb} silenced a nasty documentation warning.
\item \href{https://github.com/qwename}{\tt Yixin Zhang} fixed an integer overflow check.
\item \href{https://github.com/Bosswestfalen}{\tt Bosswestfalen} merged two iterator classes into a smaller one.
\item \href{https://github.com/Daniel599}{\tt Daniel599} helped to get Travis execute the tests with Clang\textquotesingle{}s sanitizers.
\item \href{https://github.com/vjon}{\tt Jonathan Lee} fixed an example in the R\+E\+A\+D\+ME file.
\item \href{https://github.com/gnzlbg}{\tt gnzlbg} supported the implementation of user-\/defined types.
\item \href{https://github.com/qis}{\tt Alexej Harm} helped to get the user-\/defined types working with Visual Studio.
\item \href{https://github.com/jaredgrubb}{\tt Jared Grubb} supported the implementation of user-\/defined types.
\item \href{https://github.com/EnricoBilla}{\tt Enrico\+Billa} noted a typo in an example.
\item \href{https://github.com/horenmar}{\tt Martin Hořeňovský} found a way for a 2x speedup for the compilation time of the test suite.
\item \href{https://github.com/ukhegg}{\tt ukhegg} found proposed an improvement for the examples section.
\item \href{https://github.com/rswanson-ihi}{\tt rswanson-\/ihi} noted a typo in the R\+E\+A\+D\+ME.
\item \href{https://github.com/stanmihai4}{\tt Mihai Stan} fixed a bug in the comparison with {\ttfamily nullptr}s.
\item \href{https://github.com/tusharpm}{\tt Tushar Maheshwari} added \href{https://github.com/sakra/cotire}{\tt cotire} support to speed up the compilation.
\item \href{https://github.com/TedLyngmo}{\tt Ted\+Lyngmo} noted a typo in the R\+E\+A\+D\+ME, removed unnecessary bit arithmetic, and fixed some {\ttfamily -\/\+Weffc++} warnings.
\item \href{https://github.com/krzysztofwos}{\tt Krzysztof Woś} made exceptions more visible.
\item \href{https://github.com/ftillier}{\tt ftillier} fixed a compiler warning.
\item \href{https://github.com/tinloaf}{\tt tinloaf} made sure all pushed warnings are properly popped.
\item \href{https://github.com/Fytch}{\tt Fytch} found a bug in the documentation.
\item \href{https://github.com/Type1J}{\tt Jay Sistar} implemented a Meson build description.
\item \href{https://github.com/HenryRLee}{\tt Henry Lee} fixed a warning in I\+CC and improved the iterator implementation.
\item \href{https://github.com/vthiery}{\tt Vincent Thiery} maintains a package for the Conan package manager.
\item \href{https://github.com/koemeet}{\tt Steffen} fixed a potential issue with M\+S\+VC and {\ttfamily std\+::min}.
\item \href{https://github.com/Chocobo1}{\tt Mike Tzou} fixed some typos.
\item \href{https://github.com/amrcode}{\tt amrcode} noted a missleading documentation about comparison of floats.
\item \href{https://github.com/olegendo}{\tt Oleg Endo} reduced the memory consumption by replacing {\ttfamily $<$iostream$>$} with {\ttfamily $<$iosfwd$>$}.
\item \href{https://github.com/dan-42}{\tt dan-\/42} cleaned up the C\+Make files to simplify including/reusing of the library.
\item \href{https://github.com/himikof}{\tt Nikita Ofitserov} allowed for moving values from initializer lists.
\item \href{https://github.com/wincent}{\tt Greg Hurrell} fixed a typo.
\item \href{https://github.com/DmitryKuk}{\tt Dmitry Kukovinets} fixed a typo.
\item \href{https://github.com/kbthomp1}{\tt kbthomp1} fixed an issue related to the Intel O\+SX compiler.
\item \href{https://github.com/daixtrose}{\tt Markus Werle} fixed a typo.
\item \href{https://github.com/WebProdPP}{\tt Web\+Prod\+PP} fixed a subtle error in a precondition check.
\item \href{https://github.com/leha-bot}{\tt Alex} noted an error in a code sample.
\item \href{https://github.com/tdegeus}{\tt Tom de Geus} reported some warnings with I\+CC and helped fixing them.
\item \href{https://github.com/pjkundert}{\tt Perry Kundert} simplified reading from input streams.
\item \href{https://github.com/sonulohani}{\tt Sonu Lohani} fixed a small compilation error.
\item \href{https://github.com/jseward}{\tt Jamie Seward} fixed all M\+S\+VC warnings.
\item \href{https://github.com/eld00d}{\tt Nate Vargas} added a Doxygen tag file.
\item \href{https://github.com/pvleuven}{\tt pvleuven} helped fixing a warning in I\+CC.
\item \href{https://github.com/crea7or}{\tt Pavel} helped fixing some warnings in M\+S\+VC.
\item \href{https://github.com/jseward}{\tt Jamie Seward} avoided unneccessary string copies in {\ttfamily find()} and {\ttfamily count()}.
\end{DoxyItemize}

Thanks a lot for helping out! Please \href{mailto:mail@nlohmann.me}{\tt let me know} if I forgot someone.

\subsection*{Used third-\/party tools}

The library itself contains of a single header file licensed under the M\+IT license. However, it is built, tested, documented, and whatnot using a lot of third-\/party tools and services. Thanks a lot!


\begin{DoxyItemize}
\item \href{http://lcamtuf.coredump.cx/afl/}{\tt {\bfseries American fuzzy lop}} for fuzz testing
\item \href{https://www.appveyor.com}{\tt {\bfseries App\+Veyor}} for \href{https://ci.appveyor.com/project/nlohmann/json}{\tt continuous integration} on Windows
\item \href{http://astyle.sourceforge.net}{\tt {\bfseries Artistic Style}} for automatic source code identation
\item \href{https://github.com/sbs-ableton/benchpress}{\tt {\bfseries benchpress}} to benchmark the code
\item \href{https://github.com/philsquared/Catch}{\tt {\bfseries Catch}} for the unit tests
\item \href{http://clang.llvm.org}{\tt {\bfseries Clang}} for compilation with code sanitizers
\item \href{https://cmake.org}{\tt {\bfseries Cmake}} for build automation
\item \href{https://www.codacy.com}{\tt {\bfseries Codacity}} for further \href{https://www.codacy.com/app/nlohmann/json}{\tt code analysis}
\item \href{https://coveralls.io}{\tt {\bfseries Coveralls}} to measure \href{https://coveralls.io/github/nlohmann/json}{\tt code coverage}
\item \href{https://scan.coverity.com}{\tt {\bfseries Coverity Scan}} for \href{https://scan.coverity.com/projects/nlohmann-json}{\tt static analysis}
\item \href{http://cppcheck.sourceforge.net}{\tt {\bfseries cppcheck}} for static analysis
\item \href{https://github.com/jarro2783/cxxopts}{\tt {\bfseries cxxopts}} to let benchpress parse command-\/line parameters
\item \href{http://www.stack.nl/~dimitri/doxygen/}{\tt {\bfseries Doxygen}} to generate \href{https://nlohmann.github.io/json/}{\tt documentation}
\item \href{https://github.com/rstacruz/git-update-ghpages}{\tt {\bfseries git-\/update-\/ghpages}} to upload the documentation to gh-\/pages
\item \href{https://github.com/skywinder/github-changelog-generator}{\tt {\bfseries Github Changelog Generator}} to generate the https\+://github.com/nlohmann/json/blob/develop/\+Change\+Log.\+md \char`\"{}\+Change\+Log\char`\"{}
\item \href{http://llvm.org/docs/LibFuzzer.html}{\tt {\bfseries lib\+Fuzzer}} to implement fuzz testing for O\+S\+S-\/\+Fuzz
\item \href{https://github.com/google/oss-fuzz}{\tt {\bfseries O\+S\+S-\/\+Fuzz}} for continuous fuzz testing of the library
\item \href{https://github.com/nlohmann/json/blob/develop/doc/scripts/send_to_wandbox.py}{\tt {\bfseries send\+\_\+to\+\_\+wandbox}} to send code examples to \href{http://melpon.org/wandbox}{\tt Wandbox}
\item \href{https://travis-ci.org}{\tt {\bfseries Travis}} for \href{https://travis-ci.org/nlohmann/json}{\tt continuous integration} on Linux and mac\+OS
\item \href{http://valgrind.org}{\tt {\bfseries Valgrind}} to check for correct memory management
\item \href{http://melpon.org/wandbox}{\tt {\bfseries Wandbox}} for \href{http://melpon.org/wandbox/permlink/4NEU6ZZMoM9lpIex}{\tt online examples}
\end{DoxyItemize}

\subsection*{Projects using J\+S\+ON for Modern C++}

The library is currently used in Apple mac\+OS Sierra and i\+OS 10. I am not sure what they are using the library for, but I am happy that it runs on so many devices.

\subsection*{Notes}


\begin{DoxyItemize}
\item The code contains numerous debug {\bfseries assertions} which can be switched off by defining the preprocessor macro {\ttfamily N\+D\+E\+B\+UG}, see the \href{http://en.cppreference.com/w/cpp/error/assert}{\tt documentation of {\ttfamily assert}}. In particular, note \href{https://nlohmann.github.io/json/classnlohmann_1_1basic__json_a2e26bd0b0168abb61f67ad5bcd5b9fa1.html#a2e26bd0b0168abb61f67ad5bcd5b9fa1}{\tt {\ttfamily operator\mbox{[}\mbox{]}}} implements {\bfseries unchecked access} for const objects\+: If the given key is not present, the behavior is undefined (think of a dereferenced null pointer) and yields an \href{https://github.com/nlohmann/json/issues/289}{\tt assertion failure} if assertions are switched on. If you are not sure whether an element in an object exists, use checked access with the \href{https://nlohmann.github.io/json/classnlohmann_1_1basic__json_a674de1ee73e6bf4843fc5dc1351fb726.html#a674de1ee73e6bf4843fc5dc1351fb726}{\tt {\ttfamily at()} function}.
\item As the exact type of a number is not defined in the \href{http://rfc7159.net/rfc7159}{\tt J\+S\+ON specification}, this library tries to choose the best fitting C++ number type automatically. As a result, the type {\ttfamily double} may be used to store numbers which may yield \href{https://github.com/nlohmann/json/issues/181}{\tt {\bfseries floating-\/point exceptions}} in certain rare situations if floating-\/point exceptions have been unmasked in the calling code. These exceptions are not caused by the library and need to be fixed in the calling code, such as by re-\/masking the exceptions prior to calling library functions.
\item The library supports {\bfseries Unicode input} as follows\+:
\begin{DoxyItemize}
\item Only {\bfseries U\+T\+F-\/8} encoded input is supported which is the default encoding for J\+S\+ON according to \href{http://rfc7159.net/rfc7159#rfc.section.8.1}{\tt R\+FC 7159}.
\item Other encodings such as Latin-\/1, U\+T\+F-\/16, or U\+T\+F-\/32 are not supported and will yield parse errors.
\item \href{http://www.unicode.org/faq/private_use.html#nonchar1}{\tt Unicode noncharacters} will not be replaced by the library.
\item Invalid surrogates (e.\+g., incomplete pairs such as {\ttfamily \textbackslash{}u\+D\+E\+AD}) will yield parse errors.
\item The strings stored in the library are U\+T\+F-\/8 encoded. When using the default string type ({\ttfamily std\+::string}), note that its length/size functions return the number of stored bytes rather than the number of characters or glyphs.
\end{DoxyItemize}
\item The code can be compiled without C++ {\bfseries runtime type identification} features; that is, you can use the {\ttfamily -\/fno-\/rtti} compiler flag.
\item {\bfseries Exceptions} are used widely within the library. They can, however, be switched off with either using the compiler flag {\ttfamily -\/fno-\/exceptions} or by defining the symbol {\ttfamily J\+S\+O\+N\+\_\+\+N\+O\+E\+X\+C\+E\+P\+T\+I\+ON}. In this case, exceptions are replaced by an {\ttfamily abort()} call.
\item By default, the library does not preserve the {\bfseries insertion order of object elements}. This is standards-\/compliant, as the \href{https://tools.ietf.org/html/rfc7159.html}{\tt J\+S\+ON standard} defines objects as \char`\"{}an unordered collection of zero or more name/value pairs\char`\"{}. If you do want to preserve the insertion order, you can specialize the object type with containers like \href{https://github.com/Tessil/ordered-map}{\tt {\ttfamily tsl\+::ordered\+\_\+map}} (\href{https://github.com/nlohmann/json/issues/546#issuecomment-304447518}{\tt integration}) or \href{https://github.com/nlohmann/fifo_map}{\tt {\ttfamily nlohmann\+::fifo\+\_\+map}} (\href{https://github.com/nlohmann/json/issues/485#issuecomment-333652309}{\tt integration}).
\end{DoxyItemize}

\subsection*{Execute unit tests}

To compile and run the tests, you need to execute


\begin{DoxyCode}
$ mkdir build
$ cd build
$ cmake ..
$ cmake --build .
$ ctest
\end{DoxyCode}


For more information, have a look at the file \href{https://github.com/nlohmann/json/blob/master/.travis.yml}{\tt .travis.\+yml}. 